% Task 1
\documentclass[12pt,a4paper,english]{extarticle}
\usepackage[T1]{fontenc}
\usepackage[utf8]{inputenc}
\usepackage{fourier}
\usepackage{geometry}
\geometry{verbose,tmargin=2.2cm,bmargin=2cm,lmargin=2.2cm,rmargin=2cm}
\usepackage{float}
\usepackage{textcomp}
\usepackage{amsmath}
\usepackage{stackrel}
\usepackage{graphicx}
\usepackage{esint}
\usepackage{tikz}
\usepackage{array}
\usepackage{multirow}
\usetikzlibrary{matrix,calc}

\makeatletter

\providecommand{\tabularnewline}{\\}

\usepackage{fancyhdr}
\usepackage{lscape}
\usepackage{amssymb}
\pagestyle{fancy}
\lhead{Electronica III - 22.13}
\chead{TPL3}
\rhead{ITBA}
\renewcommand{\headrulewidth}{1pt}
\renewcommand{\footrulewidth}{1pt}

\makeatother

\usepackage[english]{babel}

\begin{document}

\section*{Task 1}

In this section, a state machine will be 
developed for controlling the switching
on and off of two pumps, to fill a tank.
The are controlled by two sensors from 
the upper part of the tank (S) and the 
lower part of the tank (I). The actions
to take are as follows:

\begin{itemize}
    \item Tank full: S = I = 1 - Pumps OFF
    \item Tank empty: S = I = 0 - Pumps ON
    \item Half full tank: S = 0 \& I = 1 - Pumps alternate 
\end{itemize}

With this in mind, a Moore machine is 
developed as follows.

% DIBUJO DE MOORE 1A

Using two bits to asign the states, a table 
of transitions is made, as shown below.

% DIBUJO DE LA TABLA 1A

From the table, using Karnaugh's maps the 
functions for the state variables and the 
two pumps outputs are made as shown below.

% DIAGRAMAS DE KARNAUGH 1A

Finally, the state machine is implemented using 
D Flip Flops:

% CIRCUITO MOORE 1A

On the other side, the same system is implemented 
now using a Mealy state machine, as shown below.

% DIBUJO DE MEALY 1B

Notice that the direct connection between the 
input and the pumps outputs reduces the number 
of states from four to two, in comparison with 
the Moore machine.
Using one bit for the states, a table of transitions
is made as follows.

% TABLA 1B

Using again Karnaugh's maps, the functiosn for
the state variable and the two pumps outputs are
made as shown below.

% DIAGRAMAS DE KARNAUGH 1B

Finally, the state machine is implemented using
one D Flip Flop.

% CIRCUITO MEALY 1B

\end{document}