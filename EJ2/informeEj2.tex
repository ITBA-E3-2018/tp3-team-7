% Task 2
\input{baseHoja.tex}
\begin{document}

\section*{Task 2}

In this section, the objetive is to recognize a 
sequence of 4 bits that come in a synchronized
way. If the sequence is recognized, an output 
is turned on. Using a Moore's state machine,
the resulting diagram is as shown below.

% DIAGRAMA MOORE 2A

Notice that when the sequence is recognized, 
the machine needs to be reseted to detect a 
new combination. With the diagram, the following transition table 
is made.

% TABLA 2A

Using Karnaugh's maps, the functions for
the diferent states and the output are made
as follows.

% KARNAUGH 2A

With the functions, the state machine is 
implemented using 3 D Flip Flops as shown below.

% CIRCUITO MOORE 2A

Now, the same system is implemented using a Mealy's 
state machine, wich resulting diagram is shown below.

% DIAGRAMA MEALY 2B

Using the diagram, a table with the state 
transitions is made.

% TABLA 2B

Using Karnaugh's maps, the functions for the 
states and the output are made below.

% KARNAUGH 2B

With the defined functions, the state machine 
is implemented with 2 D Flip Flops. In this case
is used one less flip flop, and the machine can
used again without reset it.

% CIRCUITO MEALY 2B

\end{document}