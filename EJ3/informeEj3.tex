% Task 3
\input{baseHoja.tex}
\begin{document}

\section*{Task 3}

In this section, is implemented a Moore's state
machine already defined, as shown below.

% FOTO DE LA DE MOORE 3A

Using the diagram, the following table of transitions 
is made.

% TABLA DE MOORE 3A

With the transitions, using Karnaugh's maps, the 
functions for the states and the output are made
as shown below.

% KARNAUGH 3A

With the functions, the state machine is implemented 
using two D Flip Flops as follows. 

% CIRCUIT MOORE 3A

Now the same system is implemented using a Mealy's 
state machine, wich resulting diagram is shown below.

% DIAGRAMA MEALY 3B

Notice that it requires one less state than Moore's
machine because of the direct connection of the 
from the input to the output. The following 
transition table is made using the diagram.

% TABLA 3B

With the table, using Karnaugh's maps, are made 
the functions for the states and the output.

% KARNAUGH 3B

With the defined functions, the state machine is 
implemented using one D Flip Flop as shown below.

% CIRCUITO MEALY 3B

Since the internal logic works with 3.3V power supply, 
and the external signals work with 5V, level shifters 
are implemented using BJT transistors. For adapting
the inputs of CLK and W, the circuit is shown 
below. 

% LEVEL SHIFTER INPUTS

And for the outputs (Moore and Mealy machines) 
the driver circuit is as shown below.

% CIRCUITO DE LED

The design of the circuits with the corresponding 
calculations are included in the $Annex$.

\newpage

\section*{Annex}
\subsection*{Level shifter for inputs}

\subsection*{Level shifter for outputs}

\subsection*{Driver for output leds}

\end{document}