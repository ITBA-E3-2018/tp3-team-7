% ANNEX3
\input{baseHoja.tex}
\input{BaseKarnaugh3.tex}
\begin{document}

\subsection*{Karnaugh's maps for Task 3}
For the Moore's state machine implementation:


\begin{figure}[H]
    \begin{center}
         \begin{Karnaughvuit}
            \minterms{5,6}
             \maxterms{0,1,2,4}
             \indeterminats{3,7}
            \implicant{7}{6}{red}
            \implicant{5}{7}{green}
         \end{Karnaughvuit}
         \begin{Karnaughvuit}
            \minterms{2,4,7}
            \maxterms{0,1,3,5,6}
            \implicantsol{2}{red}
            \implicantsol{4}{green}
            \implicantsol{7}{blue}
         \end{Karnaughvuit}
         \caption{Maps for $Y_2$ (left) and $Y_1$ (right) functions.}
    \end{center}
    \end{figure}
Where $Y_2 = W \cdot y_1 + W \cdot y_2$, and $Y_1 = W \cdot \overline{y_2} \cdot \overline{y_1}$. 
From the transitions table, it is simple to see that $Z = y_1$.

And for the Mealy's state machine implementation:

\begin{figure}[H]
    \begin{center}
     \begin{Karnaughquatre}
         \minterms{1,3}
        \maxterms{0,2}
        \implicant{1}{3}{red}
     \end{Karnaughquatre}
     \begin{Karnaughquatre}
        \minterms{1}
       \maxterms{0,2,3}
       \implicantsol{1}{red}
    \end{Karnaughquatre}
     \caption{Maps for $Y$ (left) and $Z$ (right)}
    \end{center}
\end{figure}
Where from the left map $Y = W$, and from the right
table $Z = \overline{y} \cdot W$.
\end{document}
