% Task 3
\newpage

\section*{Task 3}

In this section, is implemented a Moore's state
machine already defined, as shown below.

\begin{figure}[H]
    \begin{centering}
    \includegraphics[width=0.8\textwidth]{data/Graficos3/3a_fsm.png}
    \par\end{centering}
    \caption{Moore state machine diagram}
\end{figure}

Using the diagram, the following table of transitions 
is made.

\begin{figure}[H]
\begin{center}
\begin{tabular}{|c|c|c|c||c|}
    \hline 
    \multicolumn{2}{|c|}{Estado Actual} & \multicolumn{2}{c||}{Estado Siguiente} & \multicolumn{1}{c|}{Salidas}\tabularnewline
    \hline 
    \hline 
    \multirow{2}{*}{} & \multirow{2}{*}{y2 - y1} & \multicolumn{2}{c||}{W} & \multirow{2}{*}{Z}\tabularnewline
    \cline{3-4} 
     &  & \multicolumn{1}{c|}{0} & \multicolumn{1}{c||}{1} & \tabularnewline
    \hline 
    A & 00 & A & B & 0\tabularnewline
    \hline 
    B & 01 & A & C & 1\tabularnewline
    \hline 
    C & 10 & A & C & 0\tabularnewline
    \hline 
    \end{tabular}
    \caption{Moore state machine - Transitions}
\end{center}
\end{figure}
With the transitions, using Karnaugh's maps (see resolution in $Annex$), the 
functions for the states and the output result as follows: $Y_2 = W \cdot y_1 + W \cdot y_2$, and $Y_1 = W \cdot \overline{y_2} \cdot \overline{y_1}$. 
\newpage
From the transitions table, it is simple to see that $Z = y_1$.
With the functions, the state machine is implemented 
using two D Flip Flops as follows. 

\begin{figure}[H]
    \begin{centering}
    \includegraphics[width=0.7\textwidth]{data/Graficos3/3a_Compuertas_Moore.png}
    \par\end{centering}
    \caption{Moore state machine - Circuit implementation}
\end{figure}

Now the same system is implemented using a Mealy's 
state machine, wich resulting diagram is shown below.

\begin{figure}[H]
    \begin{centering}
    \includegraphics[width=0.6\textwidth]{data/Graficos3/3b_fsm.png}
    \par\end{centering}
    \caption{Mealy state machine diagram}
\end{figure}

Notice that it requires one less state than Moore's
machine because of the direct connection of the 
from the input to the output. 
The following 
transition table is made using the diagram.

\begin{figure}[H]
    \begin{center}
        \begin{tabular}{|c|c|c|c||c|c|}
            \hline 
            \multicolumn{2}{|c|}{Estado Actual} & \multicolumn{2}{c||}{Estado Siguiente} & \multicolumn{2}{c|}{Salidas}\tabularnewline
            \hline 
            \hline 
            \multirow{2}{*}{} & \multirow{2}{*}{y} & \multicolumn{2}{c||}{W} & \multicolumn{2}{c|}{Z}\tabularnewline
            \cline{3-6} 
             &  & \multicolumn{1}{c|}{0} & \multicolumn{1}{c||}{1} & W=0 & W=1\tabularnewline
            \hline 
            A & 0 & A & B & 0 & 1\tabularnewline
            \hline 
            B & 1 & A & B & 0 & 0\tabularnewline
            \hline 
            \end{tabular}
        \caption{Mealy state machine - Transitions}
    \end{center}
\end{figure}

With the table, using Karnaugh's maps (see resolution in $Annex$), are made 
the functions for the states and the output as follows: $Y = W$, and $Z = \overline{y} \cdot W$.
With the defined functions, the state machine is 
implemented using one D Flip Flop as shown below.

\begin{figure}[H]
    \begin{centering}
    \includegraphics[width=0.5\textwidth]{data/Graficos3/3b_Compuertas_Mealy.png}
    \par\end{centering}
    \caption{Mealy state machine - Circuit implementation}
\end{figure}
Since the internal logic works with 3.3V power supply, 
and the external signals work with 5V, level shifters 
are implemented using BJT transistors. For adapting
the inputs of CLK and W, the circuits are shown 
below. 

\begin{figure}[H]
    \begin{centering}
    \includegraphics[width=0.55\textwidth]{data/Graficos3/CLK_Driver.png}
    \par\end{centering}
    \caption{Level shifter for CLK from 5V to 3.3V (VCC)}
\end{figure}

\begin{figure}[H]
    \begin{centering}
    \includegraphics[width=0.55\textwidth]{data/Graficos3/W_Driver.png}
    \par\end{centering}
    \caption{Level shifter for W from 5V to 3.3V (VCC). The inverting gate is to compense the transistor logic inversion.}
\end{figure}

And for the outputs (Moore and Mealy machines) 
the driver circuit is as shown below.

\begin{figure}[H]
    \begin{centering}
    \includegraphics[width=0.4\textwidth]{data/Graficos3/LED_Driver.png}
    \par\end{centering}
    \caption{Driver for LED output.}
\end{figure}

The design of the circuit is described in the $Annex$.
\subsection *{Verilog tests}
This task was tested using verilog tests. Here we show the results of Moore and Mealy's implementations.
For extra support, it was always simulated on Proteus.

\begin{figure}[H]
    \begin{centering}
    \includegraphics[width=0.7\textwidth]{data/Graficos3/3b.png}
    \par\end{centering}
    \caption{Moore state machine - Verilog test results}
\end{figure}

\begin{figure}[H]
    \begin{centering}
    \includegraphics[width=0.7\textwidth]{data/Graficos3/3a.png}
    \par\end{centering}
    \caption{Mealy state machine - Verilog test results}
\end{figure}






